Besides its position, every rigid body in 3D space may have an orientation. This introduces an
additional three degrees of freedom for each stationary body (six degrees of freedom if
angular velocity is included). Unfortunately, there is no similarly neat representation of
orientation as cartesian coordinates are for the position. The most common schemes describe
orientation in terms of a rotation operation which transforms a vector in the body's local
coordinates into world coordinates (or vice-versa). There is, however, no obvious best
answer to the question how to represent this rotation: the three most common representations,
described here, all have their advantages and disadvantages.

{\em Euler angles} are probably the most intuitive representation of a 3D rotation, describing
it as a series of three rotations about different axes. These axes are fixed by convention, so it
suffices to specify the three angles of rotation. However, this scheme has a number of drawbacks
which are extensively discussed in the literature: amongst other things, it is possible that
rotation about an axis freezes during an animation (``Gimbal lock'') unless all special cases
are handled very carefully.

{\em Rotation matrices} are commonly used because they are well understood, easily generalize
to other dimensions and allow efficient combination with other linear transformations (scaling
and shearing~-- translation may also be included if homogeneous coordinates are employed).
Unfortunately, many operations required for animation of rotations are awkward to implement
since this representation uses nine numbers (a $3\times3$ matrix) to represent three degrees
of freedom, thus introducing six additional side-conditions.

{\em Quaternions} are a popular alternative to the two previous schemes, and they are also
used extensively in this project. Mathematically, they can be regarded as numbers with one real
part and three distinct imaginary parts:
\begin{equation}
\mathbf{q} = w + x\mathbf{i} + y\mathbf{j} + z\mathbf{k}
\end{equation}
where $w$, $x$, $y$ and $z$ are real and $\mathbf{i}$, $\mathbf{j}$ and $\mathbf{k}$ satisfy
\begin{equation}
\mathbf{i}^2 = \mathbf{j}^2 = \mathbf{k}^2 = \mathbf{i}\mathbf{j}\mathbf{k} = -1.
\end{equation}
From this follows that
$\mathbf{i}\mathbf{j} = -\mathbf{j}\mathbf{i} = \mathbf{k}$ and
$\mathbf{j}\mathbf{k} = -\mathbf{k}\mathbf{j} = \mathbf{i}$ and
$\mathbf{k}\mathbf{i} = -\mathbf{i}\mathbf{k} = \mathbf{j}$.
Note that multiplication is not commutative.

We will also need the conjugate and the inverse of a quaternion. In analogy to
complex numbers, these are given respectively by
\begin{eqnarray}
\mathbf{q}^\ast & = & w - x\mathbf{i} - y\mathbf{j} - z\mathbf{k}\\
\mathbf{q}^{-1} & = & \frac{\mathbf{q}^\ast}{||\mathbf{q}||}
\end{eqnarray}

The complex constants $\mathbf{i}$, $\mathbf{j}$ and $\mathbf{k}$ are required for the algebra
only, therefore we can represent a quaternion as four numbers $(w,x,y,z)$. It turns out that
quaternions neatly represent a rotation in 3D space (similarly to the way that ordinary complex
numbers represent a rotation in the 2D Argand diagram) if we require their magnitude to be
equal to one:
\begin{equation}
||\mathbf{q}||^2 = w^2 + x^2 + y^2 + z^2 = 1.
\end{equation}
This condition reduces the number of degrees of freedom to three, as required.

Every unit quaternion now represents a rotation of angle $\theta$ about an arbitrary axis.
If the axis passes through the origin and has a direction given by the vector
$\mathbf{a} = (a_1, a_2, a_3)$ with $||\mathbf{a}|| = 1$, the quaternion describing this rotation is
\begin{equation}
\mathbf{q} = \cos\left(\frac{\theta}{2}\right) + (a_1\mathbf{i} + a_2\mathbf{j} +
    a_3\mathbf{k}) \sin\left(\frac{\theta}{2}\right).
\end{equation}
It can easily be verified that this quaternion always has unit magnitude.
We shall assume throughout this project that the rotation thus described is clockwise (as seen
when looking in the direction of the vector $\mathbf{a}$), i.e. that it is given by the
``right-hand rule''.

To rotate a vector $\mathbf{v} = (v_1, v_2, v_3)$ by a quaternion $\mathbf{q}$, we first
convert it into a quaternion $\tilde{\mathbf{v}} = v_1\mathbf{i} + v_2\mathbf{j} + v_3\mathbf{k}$
and then calculate the quaternion product
\begin{equation}
\label{quatTransform}
\tilde{\mathbf{v}}^\prime = \mathbf{q}\tilde{\mathbf{v}}\mathbf{q}^{-1}
\end{equation}

By multiplying out this formula, we find that the real part of the result is always zero, and
that the rotated vector $\mathbf{v}^\prime = (v_1^\prime, v_2^\prime, v_3^\prime)$ is contained
in the three complex parts of the quaternion product $\tilde{\mathbf{v}}^\prime$.

Some authors, notably Shoemake, choose to define the product in equation~\ref{quatTransform} in
the reverse order. The choice is a matter of convention, since it merely changes the effect
of this operation from being a clockwise to a counter-clockwise rotation. We chose the clockwise
convention because it is consistent with the usual definition of the angular velocity vector
in physics.

Observe that under this convention, if $\mathbf{q}$ is itself a product of quaternions, the
result of this transformation is the same as if the rotations corresponding to the factors
of $\mathbf{q}$ had been applied starting with the rightmost factor, and continuing from
right to left. To verify that this is the case, the identity
$(q_1 q_2)^{-1} = q_2^{-1} q_1^{-1}$ is useful.
