\documentclass{article}
\textwidth 12.5cm
\textheight 23cm
\begin{document}
\pagestyle{empty}
{\hfill Martin Kleppmann}

{\hfill Corpus Christi College}

{\hfill mk428}

\vspace{1.5cm}
\centerline{\Large\bf Rigid body simulation for 3D character animation}
\vspace{0.5cm}
\centerline{\large\bf Progress report}
\vspace{0.5cm}
\centerline{2 February 2006}

\vspace{1.5cm}
\begin{tabular}{ll}
{\bf Project Originator:} & Martin Kleppmann\\
{\bf Project Supervisor:} & Dr Neil Dodgson\\
{\bf Director of Studies:} & Dr David Greaves\\
{\bf Overseers:} & Dr Graham Titmus and Dr Markus Kuhn\\
\end{tabular}
\vspace{1.5cm}

16 weeks into the project, good progress has been made. The problem of adapting a standard
rigid body simulation to the articulated bodies of 3D characters turned out to be much harder than
expected, but nevertheless the project is running only about two weeks behind schedule.

The first four milestones of the project went smoothly, incurring only marginal delays. From early
December onwards, when the core dynamics simulation was to be implemented, I suddenly found myself
facing difficulties. This was particularly due to a lack of appropriate literature~--
all sources I could find were either too simplified (e.g.\ reducing the problem to two
dimensions), or overly terse and vague; other books used obscure notation which was a nightmare
to come to terms with; also some important sources contained a large number of errors.  I ended
up having to work out most of the mathematical background of the algorithms from first principles,
which took considerably longer than the allotted time, and consumed most of the time originally
reserved as a holiday in the schedule.

On the positive side, such detailed exposure to the mathematical background led to some interesting
discoveries. I have not yet had the time to conduct a very thorough literature survey, but I
have not yet come across some of these algorithms in my references, and hence have some grounds to
believe that these may include publishable results. The discoveries include a more accurate way of
numerically solving differential equations involving quaternions, a simpler algorithm for computing
contact forces between resting bodies, and a unified scheme for handling different types of
collisions.

Some adjustments were made to the original project plan. Parts of the dissertation writing process
have been moved forward in the schedule and have already been completed. Simulating static and
sliding friction forces has been declared an extension (previously a core requirement) because it
is very difficult and there appears to be hardly any literature on the subject. A shift of
emphasis favours the simulation of interactions between multiple passive bodies over muscular
forces, because the latter are difficult to control automatically.
\end{document}
