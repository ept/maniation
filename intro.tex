\section{Motivation}

3D animated graphics have become an everyday part of our lives through films, advertising and
computer games. Recent years have seen not only the production of feature-length animated films,
but also the widespread use of animation in combination with traditionally shot footage. The
techniques are now so refined that computer generated and recorded pictures are sometimes
indistinguishable even to experts, thus opening up a whole range of new artistic possibilities.

Currently computer animation still involves a large amount of manual effort, despite commonly
being termed ``computer generated''. In large-budget film productions every single frame is
edited by hand, and many animations are completely hand-crafted. There are artistic reasons for
this~-- sometimes an animation is deliberately required to be physically impossible for a special
effect~-- and also economic reasons (animators are cheaper to hire than computer scientists).
However, as the complexity of animated scenes increases, manual animation becomes unfeasible.
Large crowds, fluids, cloths and some types of deformable bodies are therefore commonly animated
using simulation techniques.

This project aims to address character animation, i.e.\ the movement of skeleton-based deformable
bodies like humans and animals. Certain physical systems like pendulums may also be counted to
this category. These systems have comparatively few degrees of freedom, and are mostly animated
by hand. As we shall see in this project, they nevertheless provide plenty of scope for
interesting simulation work. Although being arguably more important, fluid dynamics is very
complicated and the physics of deformable bodies (so-called ``soft matter'') is not even
completely understood, so graphical simulations in these areas tend to resort to means which look
good but are physically meaningless. Rigid body dynamics on the other hand, which will form the
basis for physically based character animation, is well understood and has a solid theoretical
underpinning against which the simulation results can be checked. Hence the subject of this
project lends itself more to an objective evaluation than most other areas of computer graphics,
making it well suited as a Part~II project.


\section{Scope}

There are several different approaches 

games -> films -> cad


\section{Overview}

Simulation consists of <n> main blocks
- rigid body/ode
- constraints
- contact and collision handling
- collision detection
