\section{Rigid body dynamics\label{rigidBodyAppendix}}
\subsection{Useful definitions}

Given any vector $\ve{a} = (a_1, a_2, a_3)^T$, we can define its dual
tensor, written as a $3\times3$ matrix, to be
\begin{equation}
\dual{\ve{a}} = \left[\begin{array}{ccc}
    0 & -a_3 & a_2 \\ a_3 & 0 & -a_1 \\ -a_2 & a_1 & 0
    \end{array}\right]
\end{equation}
(see \cite{RHB:02,BaraffWitkin:97} and also Kalra~\cite{Kalra:95}, who defines it to be
the transpose of the expression above).
The dual allows us to re-write a vector cross product as a matrix multiplication:
\begin{equation}
\ve{a}\times\ve{b} = \dual{\ve{a}}\,\ve{b}
\end{equation}
Note that $\dual{\ve{a}}^T = -\dual{\ve{a}}$.

Let us also recall some basic identities of vector and matrix algebra~\cite{RHB:02}:
\begin{eqnarray*}
\ve{a}\times\ve{a} & = & \ve{0}\quad\mathrm{(the~null~vector)} \\
\ve{a}\times\ve{b} & = & -\ve{b}\times\ve{a} \\
\ve{a}\times(\ve{b} + \ve{c}) & = & \ve{a}\times\ve{b} + \ve{a}\times\ve{c} \\
\ve{a}\cdot\ve{b} & = & \ve{b}\cdot\ve{a} \\
\ve{a}\cdot\ve{b} & = & \ve{a}^T\,\ve{b} \\
\ve{a}\cdot(\ve{b} + \ve{c}) & = & \ve{a}\cdot\ve{b} + \ve{a}\cdot\ve{c} \\
\ve{a}\cdot(\ve{b}\times\ve{c}) & = & \ve{b}\cdot(\ve{c}\times\ve{a}) \\
    & = & \ve{c}\cdot(\ve{a}\times\ve{b}) \\
\ve{a}\times(\ve{b}\times\ve{c}) & = &
    \ve{b}(\ve{a}\cdot\ve{c}) - \ve{c}(\ve{a}\cdot\ve{b}) \\
(\m{A}\m{B})\m{C} & = & \m{A}(\m{B}\m{C}) \\
\m{A}(\m{B} + \m{C}) & = & \m{A}\m{B} + \m{A}\m{C} \\
(\m{A}\m{B})^T & = & \m{B}^T\m{A}^T \\
\m{A}\m{A}^{-1} = \m{A}^{-1}\m{A} & = & \m{1}\quad\mathrm{(the~identity~matrix)}
\end{eqnarray*}


\subsection{Free precession\label{correctBrettAppendix}}

This argument is modelled after~\cite{Ruf:02}. The moment of inertia $\ve{L}$ of a rigid
body is defined as
\begin{equation}
\label{correctBrett1}
\ve{L} = \m{I}\ve{\omega}
\end{equation}
where $\m{I}$ is the inertia tensor and $\ve{\omega}$ is the angular velocity vector.
Torque is the rate of change of angular momentum over time. Using the chain rule,
\begin{equation}
\label{correctBrett2}
\ve{\tau} = \dot{\ve{L}} = \dot{\m{I}}\ve{\omega} + \m{I}\dot{\ve{\omega}}
\end{equation}

We can further evaluate $\dot{\m{I}}$ by writing it as a product with a rotation matrix
$\m{R}$ and its transpose:
\begin{equation}
\label{correctBrett3}
\m{I} = \m{R}\m{I}_\mathrm{body}\m{R}^T
\end{equation}

It can be shown that such a decomposition of the inertia tensor always exists, and that
$\m{I}_\mathrm{body}$ is a diagonal, time-invariant matrix containing the moments
of inertia about the body's principal axes~\cite{Feynman:63}. Hence we obtain
\begin{equation}
\label{correctBrett4}
\dot{\m{I}} = \dot{\m{R}}\m{I}_\mathrm{body}\m{R}^T +
    \m{R}\m{I}_\mathrm{body}\frac{\diff}{\diff t}\m{R}^T
\end{equation}

Witkin~\cite{BaraffWitkin:97} derives that $\dot{\m{R}} = \dual{\ve{\omega}}\,\m{R}$
for a rotation matrix $\m{R}$ and an angular velocity vector $\ve{\omega}$.
Using this identity and taking the differential operator onto the inside of the
transpose at the end of equation~\ref{correctBrett4},
\begin{eqnarray}
\dot{\m{I}} &=& \dual{\ve{\omega}}\,\m{R}\m{I}_\mathrm{body}\m{R}^T +
    \m{R}\m{I}_\mathrm{body}(\dual{\ve{\omega}}\,\m{R})^T \nonumber\\
&=& \dual{\ve{\omega}}\,\m{R}\m{I}_\mathrm{body}\m{R}^T -
    \m{R}\m{I}_\mathrm{body}\m{R}^T\dual{\ve{\omega}} \nonumber\\
&=& \dual{\ve{\omega}}\,\m{I} - \m{I}\dual{\ve{\omega}} \label{correctBrett5}
\end{eqnarray}

Substituting equation~\ref{correctBrett5} back into~\ref{correctBrett2}:
\begin{eqnarray}
\ve{\tau} & = & \m{I}\dot{\ve{\omega}} + \dual{\ve{\omega}}\,\m{I}\ve{\omega} -
    \m{I}\dual{\ve{\omega}}\,\ve{\omega} \nonumber \\
& = & \m{I}\dot{\ve{\omega}} + \dual{\ve{\omega}}\,\m{I}\ve{\omega} \label{correctBrett6}
\end{eqnarray}

Equation~\ref{correctBrett6} corrects the similar expression in~\cite{Saunders:PhD},
page~34. This means that the angular acceleration of a rigid body is given by
\begin{equation}
\label{correctBrett7}
\dot{\ve{\omega}} = \m{I}^{-1} (\ve{\tau} - \dual{\ve{\omega}}\,\m{I}\ve{\omega}).
\end{equation}
where \ve{\tau} is the sum of all torque vectors acting on the body. This means that even if
there are no torques, its angular velocity may change if \m{I} is not diagonal (i.e.\ if
the body is somehow asymmetric). This effect is called \emph{free precession}.

In a simulation, we usually integrate over torques to find angular momentum, and then calculate
the angular velocity from the momentum in each time step using the current moment of inertia. In
this case, the angular acceleration in equation~\ref{correctBrett7} is not needed. It is required
only for purposes of computing constraint forces and torques.


\subsection{Lagrange multiplier constraints}

Both~\cite{BaraffWitkin:97} and~\cite{Saunders:PhD} omit details of how to derive the Jacobians
$\m{J}$ and $\dot{\m{J}}$ for a custom constraint. The following procedure was deduced
by the author after pondering over some source code implementing one type of constraint
(specifically equations~\ref{constrEx1J} and~\ref{constrEx1JDot} below). The source code
was kindly provided by Brett Saunders. (It should be emphasized that the code was used only
for this derivation and not copied otherwise. The implementation in this project is based
on the following derivation, not the original code.)

We start by defining a function $\ve{c}$ which evaluates to zero (or the null vector) for
all states which satisfy the constraint, and any non-zero value for all other states. 
We then calculate the first and second derivatives with respect to time, $\dot{\ve{c}}$
and $\ddot{\ve{c}}$, which must both exist.

For any sort of valid constraint, we should be able to massage both $\dot{\ve{c}}$ and
$\ddot{\ve{c}}$ into a sum of products form. Moreover, in each product in
$\dot{\ve{c}}$, at least one of the variables should be either the linear velocity of
the centre of mass of one of the bodies, or the angular velocity of one of the bodies.
Use algebra to make this `chosen variable' the rightmost variable in each product, and
evaluate the rest of the product to a single matrix.

Now observe equation~\ref{cDotAndCDDot} (page~\pageref{cDotAndCDDot}). We can easily
factor our expression for
$\dot{\ve{c}}$ into $\m{J}$ and $\dot{\ve{x}}$ (since the latter contains
the linear and angular velocities for all bodies). $\m{J}$ has the same number of rows
as there are constraints, and $6n$ columns for a system of $n$ bodies. Each of the matrices
in our expression for $\dot{\ve{c}}$ also has the same number of rows as there are
constraints, and has 3 columns. All we need to do is to find the correct horizontal position
of each of these matrices in $\m{J}$, depending on the location of the `chosen variable'
in $\dot{\ve{x}}$. Thus we obtain the matrix $\m{J}$.

Observe equation~\ref{cDotAndCDDot} again. Now we know $\ddot{\ve{c}}$, $\dot{\ve{x}}$,
$\m{J}$ and $\ddot{\ve{x}}$~-- no problem. Evaluate
$\ddot{\ve{c}} - \m{J}\ddot{\ve{x}}$, remembering that $\ddot{\ve{x}}$ is
the vector of linear accelerations and angular accelerations. The result should be an
expression which we can bring into just the same kind of sum of products form as we previously
did with $\dot{\ve{c}}$. By exactly the same procedure we factor out the linear velocities
and angular velocities (not accelerations!), thus obtaining $\dot{\m{J}}$.

In case you were in doubt, this procedure is of course executed by a human on paper
(possibly assisted by a symbolic algebra system) prior to
implementation. The simulation program will just plug numbers into the hard-coded expressions
for $\ve{c}$, $\dot{\ve{c}}$, $\m{J}$ and $\dot{\m{J}}$ during each time step
of the simulation.
