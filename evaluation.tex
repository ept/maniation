In this chapter I present some of the results obtained from performing experiments with the
simulation application described in the last three chapters.

\section{Quantitative evaluation}
\subsection{Simulation accuracy}
The quality of simulation results is hard to measure quantitatively, since simulation is usually
employed to calculate the behaviour of systems whose exact solution is not known. Therefore it is
useful to run the simulation on a system which can be characterized analytically, and to compare
the results against the theoretical values.

For this comparison I chose to simulate a \emph{gyroscope} (figure~\ref{gyroscope}). It consists
of a single rotating rigid body and a `nail' constraint (appendix~\ref{constrNail}) holding one
end of its axis in place. When gravity acts on the centre of mass of the body, the `nail' exerts
a balancing force and a torque which stop the gyroscope from falling down and causes forced
precession. The angular velocity of precession can be derived analytically~\cite{Julian:notes}
and depends only on the gravitational force, the distance between the centre of mass and the
nail, and the angular momentum of the body.

\begin{figure}
\centerline{\includegraphics{figures/gyroscope}}
\caption{Schematic drawing of a gyroscope. The disc rapidly rotates about its own axis, and
    gravity causes a slower precession movement about the vertical axis (shown as a dotted line).
    \label{gyroscope}}
\end{figure}

I set up the initial conditions for the simulation to reflect values one might find in an actual
toy gyroscope (20 revolutions per second about the gyroscope axis, one full circle of precession
in 8 seconds). I then ran the simulation for 8~s, using an average time step length of about
$2.3\cdot 10^{-4}$~s. The simulation performed one full circle of precession in 7.953~s, which
is within 0.6~\% of the theoretical value. Over the course of 8~s, the body rotated by $320.36\pi$
radians about its own axis (160.18 revolutions), which differs from the theoretical value by only
0.1~\%. These errors varied very little even in simulations using larger time steps.
The effects of nutation~\cite{Feynman:63} were small for the chosen initial conditions but may
have contributed towards the errors.

These results are very encouraging, but is interesting to also observe a different error, namely
the amount by which the constraint drifts apart. Usually this drift is compensated in the Lagrange
multiplier method so that it never manifests itself, but temporarily deactivating this
correction\footnote{by setting $k=d=0$ in equation~\ref{lagrangeEquation}.} makes the error
introduced by the ODE solver observable.

\begin{figure}
\centerline{\input{figures/errorplot}}
\caption{Errors introduced by the ODE solver for different step sizes $h$.
    Solid line: target error (difference between $O(h^4)$ and $O(h^5)$ approximations) of
    the ODE solver. Dashed line: drift of `nail' constraint in gyroscope simulation after
    8~s simulation time.\label{errorplot}}
\end{figure}

Figure~\ref{errorplot} shows by what distance the gyroscope's `nail' constraint drifted apart
after 8~s of simulation time, for a wide range of different step sizes. Since I used the
adaptive-stepsize algorithm described in section~\ref{solvingODEs}, I actually set the target
error (solid line in figure~\ref{errorplot}) and then used the average step size taken by the
solver as abscissa. There are some noteworthy features about this plot:

\begin{itemize}
\item The logarithmic axes are scaled such that one order of magnitude in the horizontal has the
    same length as five orders of magnitude in the vertical. Observe that in this scaling, the
    solid line (error per time step) is an almost perfect straight line with gradient~1. This
    shows that the error is indeed an $O(h^5)$ function of the step size, as expected.
\item Over a wide range of step sizes, the plot of the total accumulated error is parallel to the
    solid line. This means the total error is also $O(h^5)$, which is in fact better than
    expected: although the approximation in each time step is $O(h^5)$, the number of steps
    required is inversely proportional to the step length, so one might expect a larger overall
    error. This relationship indicates that the ODE solver's target error can in fact be used
    as a reliable estimate of the overall error to within a constant factor.
\item As step sizes $h$ become very small~-- below about $3\cdot 10^{-4}$~s~-- the error in each
    step continues to scale order $O(h^5)$, but due to the huge number of steps, the accumulated
    error cannot be reduced much further. However, the errors here are in the range of nanometres,
    so they should be of little concern for computer graphics purposes.
\end{itemize}

In summary, the results for the simple gyroscope simulation inspire the confidence that the
implementation is reliable and will continue to produce realistic results for complicated systems
which lack an exact solution. They also show that the target error can conveniently be adjusted
to match the requirements, because more CPU time does~-- within sensible bounds~-- buy higher
accuracy.


\section{Qualitative evaluation}


\section{Limitations}
friction

collision response (plane/bubble approximation)

polyhedral mass properties

