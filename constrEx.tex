\section{A catalog of constraint functions\label{constraintAppendix}}

Let us consider some examples of constraints to clarify the procedure.

\subsection{Fixed point in space (`Nail')}

Let us implement a constraint which `nails' a particular point in a rigid body to a fixed point
in world space. (It's a very flexible nail, because despite fixing the position, it allows all
three modes of rotation.) Let \ve{p} be the position of the centre of mass of our rigid
body, \ve{s} the vector from the centre of mass to the point in the body we want to attach,
$\ve{\omega}$ the angular velocity of the rigid body, and \ve{t} the coordinates in world
space that we want to nail the point to. Then we can set up a very simple constraint function,
\begin{equation}
\ve{c} = \ve{p} + \ve{s} - \ve{t}
\end{equation}
which equals the null vector when $\ve{p}+\ve{s}$ and $\ve{t}$ coincide, as required.
Since this is a three-dimensional vector equation, we are actually defining three constraints
at once. Since \ve{t} does not change over time, we obtain
\begin{eqnarray}
\dot{\ve{c}} &=& \dot{\ve{p}} + \ve{\omega}\times\ve{s} \nonumber\\
&=& \dot{\ve{p}} - \dual{\ve{s}}\,\ve{\omega} \\
\ddot{\ve{c}} &=& \ddot{\ve{p}} + \dot{\ve{\omega}}\times\ve{s} +
    \ve{\omega}\times(\ve{\omega}\times\ve{s}) \nonumber\\
&=& \ddot{\ve{p}} - \dual{\ve{s}}\,\dot{\ve{\omega}} -
    \dual{(\ve{\omega}\times\ve{s})}\,\ve{\omega}
\end{eqnarray}
(cf. similar derivations in~\cite{Kalra:95}). We have already moved the `chosen variables' to
the rightmost position of each product. We will now factor $\dot{\ve{c}}$ and write out the
components of $\m{J}$ in terms of the vector components:

\begin{equation}
\label{constrEx1J}
\dot{\ve{c}} = \left[\begin{array}{ccc} 1&0&0\\0&1&0\\0&0&1 \end{array}\right]
    \dot{\ve{p}} + \left[\begin{array}{ccc}
    0 & s_3 & -s_2 \\ -s_3 & 0 & s_1 \\ s_2 & -s_1 & 0
    \end{array}\right] \ve{\omega}
\end{equation}

The two matrices in equation~\ref{constrEx1J} thus form two slices of $\m{J}$ at
the locations appropriate for $\dot{\ve{p}}$ and $\ve{\omega}$. We now continue to the
next step of the procedure:

\begin{eqnarray}
\dot{\m{J}}\dot{\ve{x}} = 
\ddot{\ve{c}} - \m{J}\ddot{\ve{x}} &=&
    \big(\ddot{\ve{p}} - \dual{\ve{s}}\,\dot{\ve{\omega}} -
    \dual{(\ve{\omega}\times\ve{s})}\,\ve{\omega}\big) -
    (\ddot{\ve{p}} - \dual{\ve{s}}\,\dot{\ve{\omega}}) \nonumber\\
& = & -\dual{(\ve{\omega}\times\ve{s})}\,\ve{\omega} \nonumber\\
& = & \left[\begin{array}{ccc} 0 &
    \omega_1 s_2 - \omega_2 s_1 &
    \omega_1 s_3 - \omega_3 s_1 \\
    \omega_2 s_1 - \omega_1 s_2 & 0 &
    \omega_2 s_3 - \omega_3 s_2 \\
    \omega_3 s_1 - \omega_1 s_3 &
    \omega_3 s_2 - \omega_2 s_3 & 0
    \end{array}\right] \ve{\omega} \nonumber\\\label{constrEx1JDot}
\end{eqnarray}

We see that here there is only one slice for $\dot{\m{J}}$; the rest of the matrix
is zero. Since $\ve{\omega}$ also occurs inside the matrix, there are actually several
alternative representations of this matrix which are equally valid.

\subsection{Ball-and-socket joint}

Two rigid bodies are attached together at a particular point in each of the bodies.
They may not separate, but all three rotational degrees of freedom are permitted. This is
a good representation e.g. of a human shoulder joint.

Let $\ve{a}$ and $\ve{b}$ be the positions of the centres of mass in the first and
second rigid body respectively. Let $\ve{s}$ be the vector from $\ve{a}$ to the 
attachment point, and $\ve{t}$ the vector from $\ve{b}$ to the attachment point.
Also let $\ve{\omega}$ be the angular velocity of the first body, and $\ve{\phi}$ that
of the second. Then our constraint function and its derivatives are:

\begin{eqnarray}
\ve{c} &=& \ve{a} + \ve{s} - \ve{b} - \ve{t} \\
\dot{\ve{c}} &=& \dot{\ve{a}} + \ve{\omega}\times\ve{s} -
    \dot{\ve{b}} - \ve{\phi}\times\ve{t} \\
\ddot{\ve{c}} &=& \ddot{\ve{a}} + \dot{\ve{\omega}}\times\ve{s} +
    \ve{\omega}\times(\ve{\omega}\times\ve{s}) -
    \ddot{\ve{b}} - \dot{\ve{\phi}}\times\ve{t} -
    \ve{\phi}\times(\ve{\phi}\times\ve{t})
\end{eqnarray}

The rest of the derivation is very similar to the previous example. We obtain four
matrix slices for \m{J} and two slices for $\dot{\m{J}}$.


\subsection{Confinement to a plane}

A particular point in a rigid body is constrained to always lie within a particular plane.
It may move arbitrarily within that plane, and the body may rotate about the constrained point.

Let \ve{b} be some arbitrary point inside the plane, \ve{n} the normal vector of
the plane, \ve{p} the centre of mass of the rigid body, \ve{\omega} its angular
velocity, and \ve{s} the vector from \ve{p} to the point in the body at which the
constraint applies. Now the plane is specified as the set of all points \ve{r} which
satisfy $(\ve{r} - \ve{b})\cdot\ve{n} = 0$.

This time, we lose only one degree of freedom, so our constraint function is a scalar. For
consistency's sake, we will write it as a one-row vector.

\begin{eqnarray}
\ve{c} &=& \ve{n}\cdot(\ve{p} + \ve{s} - \ve{b}) \nonumber\\
&=& \ve{n}\cdot\ve{p} + \ve{n}\cdot\ve{s} - \ve{n}\cdot\ve{b} \\
\dot{\ve{c}} & = & \ve{n}\cdot\dot{\ve{p}} +
    \ve{n}\cdot(\ve{\omega}\times\ve{s}) \nonumber\\
&=& \ve{n}^T\dot{\ve{p}} - \ve{n}^T \dual{\ve{s}}\,\ve{\omega} \\
\ddot{\ve{c}} &=& \ve{n}\cdot\ddot{\ve{p}} +
    \ve{n}\cdot(\dot{\ve{\omega}}\times\ve{s}) +
    \ve{n}\cdot\big(\ve{\omega}\times(\ve{\omega}\times\ve{s})\big) \nonumber\\
&=& \ve{n}^T\ddot{\ve{p}} - \ve{n}^T \dual{\ve{s}}\,\dot{\ve{\omega}} -
    \ve{n}^T \dual{(\ve{\omega}\times\ve{s})}\,\ve{\omega}
\end{eqnarray}

Writing out the components of $\dot{\ve{c}}$:
\begin{equation}
\dot{\ve{c}} = \left[\begin{array}{ccc} n_1 & n_2 & n_3 \end{array}\right] \dot{\ve{p}} +
    \left[\begin{array}{c}
    n_3 s_2 - n_2 s_3 \\
    n_1 s_3 - n_3 s_1 \\
    n_2 s_1 - n_1 s_2 \end{array}\right]^T \ve{\omega}
\end{equation}
(the second slice of $\m{J}$ has been written transposed for better readability). Again
we proceed as before:

\begin{eqnarray}
\dot{\m{J}}\dot{\ve{x}} & = & \ddot{\ve{c}} - \m{J}\ddot{\ve{x}} \nonumber\\
&=& -\ve{n}^T \dual{(\ve{\omega}\times\ve{s})}\,\ve{\omega} \nonumber\\
&=& \left[\begin{array}{c}
    n_3 \omega_1 s_2 - n_3 \omega_2 s_1 - n_2 \omega_1 s_2 + n_2 \omega_2 s_1 \\
    n_1 \omega_1 s_2 - n_1 \omega_2 s_1 - n_3 \omega_2 s_3 + n_3 \omega_3 s_2 \\
    n_2 \omega_2 s_3 - n_2 \omega_3 s_2 - n_1 \omega_3 s_1 + n_1 \omega_1 s_3
    \end{array}\right]^T \ve{\omega}
\end{eqnarray}
Again note the transposition~-- in this example, $\m{J}$ and $\dot{\m{J}}$ only
have one row.

\subsection{Relative rotation of two bodies}

In example~2, we derived the formulae for a ball-and-socket joint. How can we express other types
of joints? A good way of doing this is by augmenting the ball-and-socket constraint with
additional constraints which restrict the set of valid rotations. We can constrain the orientation
of a single rigid body by requiring that the quaternion's axis of rotation must lie within a
particular plane. This plane must pass through the centre of mass of the constrained body and
has a normal vector \ve{n}. Remembering equation~\ref{quatRotation}, we can set up the
constraint function to be
\begin{eqnarray}
\label{constrEx4C}
\ve{c} & = & \mathrm{Re}(\tilde{\ve{n}}\q{q}) \\
& = & - n_1 q_x - n_2 q_y - n_3 q_z \nonumber
\end{eqnarray}
where \q{q} is the quaternion specifying the body's orientation. Here we extract only the
real part of the quaternion product, thus we have a scalar constraint like in the previous
example. The real part of the quaternion product implicitly contains the dot product with the
axis of rotation, hence equating it to zero requires the axis to lie in the specified plane.
This reduces the number of degrees of freedom by one, provided the constraint is not redundant.

It is not completely easy to visualize what this type of constraint means. One good way to look
at it is to consider a standard two-axis joystick. If it is placed on a table, the two axes of
rotation lie in a plane parallel to the surface of this table. If you cannot turn the stick about
its own axis, the normal of the constraint plane is orthogonal to the table surface.
Don't be confused by the fact that the joystick handle happens to point in the direction of the
normal~-- any sort of obscure shape may be substituted in its place without changing the nature
of the constraint!

A more common sort of joint is the hinge, which we find in most doors, in our knees and elbows.
It allows rotation only about one particular axis. We can conveniently express it by employing two
constraints on the same body, each of which confines the axis to a plane. If the two planes are
not parallel, the axis about which rotation may occur corresponds to the line of intersection of
these two planes. If they are parallel, one of the constraints is redundant. In other words,
to make a hinge joint, we first add a ball-and-socket joint, then we find two nonparallel vectors
which are both orthogonal to the hinge axis, and use them as normal vectors for two rotatation
constraints. This reduces the original number of six degrees of freedom to one~-- the angle of the
hinge.

If we want to constrain the possible rotations between two rigid bodies, all we need to do is make
\ve{n} in equation~\ref{constrEx4C} vary with the orientation of the connected body.
Let \ve{\omega} be the angular velocity of the first body, and \ve{\phi} the angular velocity
of the second body. The constraint plane normal \ve{n} is expressed in world coordinates and
is attached to the second body (i.e. $\dot{\ve{n}} = \ve{\phi}\times\ve{n}$). \q{q}
is the quaternion expressing the rotation from the first body's frame to the second body's.
Although the expression for $\ve{c}$ looks so innocent, its two derivatives are rather
verbose, so we shall spare the reader the gruesome details of the derivation and merely state the
result:

\begin{eqnarray}
\dot{\ve{c}} &=& \m{J}\left[\begin{array}{c} \ve{\omega}\\\ve{\phi} \end{array}\right] \\
&=& \left[\begin{array}{c}
        \frac{1}{2}( n_2 q_z - n_3 q_y - n_1 q_w) \\
        \frac{1}{2}( n_3 q_x - n_1 q_z - n_2 q_w) \\
        \frac{1}{2}( n_1 q_y - n_2 q_x - n_3 q_w)
    \end{array}\right]^T \ve{\omega} +
    \left[\begin{array}{c}
        n_3 q_y - n_2 q_z \\ n_1 q_z - n_3 q_x \\ n_2 q_x - n_1 q_y
    \end{array}\right]^T \ve{\phi} \nonumber \\
\ddot{\ve{c}} &=& \m{J}
    \left[\begin{array}{c} \dot{\ve{\omega}}\\\dot{\ve{\phi}} \end{array}\right] +
    \left[\begin{array}{cccccc} \dot{J}_1 & \dot{J}_2 & \dot{J}_3 & \dot{J}_4 &
        \dot{J}_5 & \dot{J}_6 \end{array}\right]
    \left[\begin{array}{c} \ve{\omega}\\\ve{\phi} \end{array}\right] \\
\dot{J}_1 & = & \frac{3}{2} q_w (\phi_3 n_2 - \phi_2 n_3) +
                \frac{3}{2} q_y (\phi_2 n_1 - \phi_1 n_2) + \nonumber\\*&&
                \frac{3}{2} q_z (\phi_3 n_1 - \phi_1 n_3) +
                \frac{1}{4} \omega_1 (q_x n_1 + q_y n_2 + q_z n_3) \nonumber\\
\dot{J}_2 & = & \frac{3}{2} q_w (\phi_1 n_3 - \phi_3 n_1) +
                \frac{3}{2} q_x (\phi_1 n_2 - \phi_2 n_1) + \nonumber\\*&&
                \frac{3}{2} q_z (\phi_3 n_2 - \phi_2 n_3) +
                \frac{1}{4} \omega_2 (q_x n_1 + q_y n_2 + q_z n_3) \nonumber\\
\dot{J}_3 & = & \frac{3}{2} q_w (\phi_2 n_1 - \phi_1 n_2) +
                \frac{3}{2} q_y (\phi_2 n_3 - \phi_3 n_2) + \nonumber\\*&&
                \frac{3}{2} q_x (\phi_1 n_3 - \phi_3 n_1) +
                \frac{1}{4} \omega_3 (q_x n_1 + q_y n_2 + q_z n_3) \nonumber\\
\dot{J}_4 & = & q_y (\phi_1 n_2 - \phi_2 n_1) + q_z (\phi_1 n_3 - \phi_3 n_1) \nonumber\\
\dot{J}_5 & = & q_x (\phi_2 n_1 - \phi_1 n_2) + q_z (\phi_2 n_3 - \phi_3 n_2) \nonumber\\
\dot{J}_6 & = & q_x (\phi_3 n_1 - \phi_1 n_3) + q_y (\phi_3 n_2 - \phi_2 n_3) \nonumber
\end{eqnarray}
