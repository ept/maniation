\section{Rigid body dynamics\label{rigidBodyAppendix}}
\subsection{Useful definitions}

Given any vector $\mathbf{a} = (a_1, a_2, a_3)^T$, we can define its dual
tensor, written as a $3\times3$ matrix, to be
\begin{equation}
\mathbf{a}^* = \left[\begin{array}{ccc}
    0 & -a_3 & a_2 \\ a_3 & 0 & -a_1 \\ -a_2 & a_1 & 0
    \end{array}\right]
\end{equation}
(see \cite{RHB:02,BaraffWitkin:97} and also Kalra~\cite{Kalra:95}, who defines it to be
the transpose of the expression above).
The dual allows us to re-write a vector cross product as a matrix multiplication:
\begin{equation}
\mathbf{a}\times\mathbf{b} = \mathbf{a}^*\,\mathbf{b}
\end{equation}

Let us also recall some basic identities of vector and matrix algebra~\cite{RHB:02}:
\begin{eqnarray*}
\mathbf{a}\times\mathbf{a} & = & \mathbf{0} \\
\mathbf{a}\times\mathbf{b} & = & -\mathbf{b}\times\mathbf{a} \\
\mathbf{a}\cdot\mathbf{b} & = & \mathbf{a}^T\,\mathbf{b} \\
\mathbf{a}\cdot(\mathbf{b}\times\mathbf{c}) & = & \mathbf{b}\cdot(\mathbf{c}\times\mathbf{a}) \\
    & = & \mathbf{c}\cdot(\mathbf{a}\times\mathbf{b}) \\
\mathbf{a}\times(\mathbf{b}\times\mathbf{c}) & = &
    \mathbf{b}(\mathbf{a}\cdot\mathbf{c}) - \mathbf{c}(\mathbf{a}\cdot\mathbf{b}) \\
(\mathbf{A}\mathbf{B})^T & = & \mathbf{B}^T\mathbf{A}^T
\end{eqnarray*}


\subsection{Conservation of angular momentum\label{correctBrettAppendix}}

This argument is modelled after~\cite{Ruf:02}. The moment of inertia $\mathbf{L}$ of a rigid
body is defined as
\begin{equation}
\label{correctBrett1}
\mathbf{L} = \mathbf{I}\bm{\omega}
\end{equation}
where $\mathbf{I}$ is the inertia tensor and $\bm{\omega}$ is the angular velocity vector.
Torque is the rate of change of angular momentum over time. Using the chain rule,
\begin{equation}
\label{correctBrett2}
\bm{\tau} = \dot{\mathbf{L}} = \dot{\mathbf{I}}\bm{\omega} + \mathbf{I}\dot{\bm{\omega}}
\end{equation}

We can further evaluate $\dot{\mathbf{I}}$ by writing it as a product with a rotation matrix
$\mathbf{R}$ and its transpose:
\begin{equation}
\label{correctBrett3}
\mathbf{I} = \mathbf{RI}_\mathrm{body}\mathbf{R}^T
\end{equation}

It can be shown that such a decomposition of the inertia tensor always exists, and that
$\mathbf{I}_\mathrm{body}$ is a diagonal, time-invariant matrix containing the moments
of inertia about the body's principal axes~\cite{Feynman:63}. Hence we obtain
\begin{equation}
\label{correctBrett4}
\dot{\mathbf{I}} = \dot{\mathbf{R}}\mathbf{I}_\mathrm{body}\mathbf{R}^T +
    \mathbf{RI}_\mathrm{body}\frac{\mathrm{d}}{\mathrm{d}t}\mathbf{R}^T
\end{equation}

Witkin~\cite{BaraffWitkin:97} derives that $\dot{\mathbf{R}} = \bm{\omega}^*\,\mathbf{R}$
for a rotation matrix $\mathbf{R}$ and an angular velocity vector $\bm{\omega}$.
Using this identity and taking the differential operator onto the inside of the
transpose at the end of equation~\ref{correctBrett4},
\begin{eqnarray}
\dot{\mathbf{I}} &=& \bm{\omega}^*\,\mathbf{RI}_\mathrm{body}\mathbf{R}^T +
    \mathbf{RI}_\mathrm{body}(\bm{\omega}^*\,\mathbf{R})^T \nonumber\\
&=& \bm{\omega}^*\,\mathbf{RI}_\mathrm{body}\mathbf{R}^T -
    \mathbf{RI}_\mathrm{body}\mathbf{R}^T\bm{\omega}^* \nonumber\\
&=& \bm{\omega}^*\,\mathbf{I} - \mathbf{I}\bm{\omega}^* \label{correctBrett5}
\end{eqnarray}

Substituting equation~\ref{correctBrett5} back into~\ref{correctBrett2}:
\begin{eqnarray}
\bm{\tau} & = & \mathbf{I}\dot{\bm{\omega}} + \bm{\omega}^*\,\mathbf{I}\bm{\omega} -
    \mathbf{I}\bm{\omega}^*\,\bm{\omega} \nonumber \\
& = & \mathbf{I}\dot{\bm{\omega}} + \bm{\omega}^*\,\mathbf{I}\bm{\omega} \label{correctBrett6}
\end{eqnarray}

Equation~\ref{correctBrett6} corrects the similar expression in~\cite{Saunders:PhD},
page~34. From this we deduce that if we want to calculate the angular velocity of a rigid
body, the value we need to feed into our ODE solver is
\begin{equation}
\dot{\bm{\omega}} = \mathbf{I}^{-1}\bm{\tau} -
    \mathbf{I}^{-1}\bm{\omega}^*\,\mathbf{I}\bm{\omega}.
\end{equation}
This means that even if the sum of all torques acting on the body is zero, its angular
velocity may change if $\mathbf{I}$ is not diagonal (i.e. if the body is somehow asymmetric).
Remember that angular momentum is conserved if there are no external torques~\cite{Feynman:63}.
Thus in order for the simulation to conserve angular momentum correctly, the value of
$\bm{\omega}^*\,\mathbf{I}\bm{\omega}$ must always be subtracted from the torques.


\subsection{Lagrange multiplier constraints}

Both~\cite{BaraffWitkin:97} and~\cite{Saunders:PhD} omit details of how to derive the Jacobians
$\mathbf{J}$ and $\dot{\mathbf{J}}$ for a custom constraint. The following procedure was deduced
by the author after pondering over some source code implementing one type of constraint
(specifically equations~\ref{constrEx1J} and~\ref{constrEx1JDot} below). The source code
was kindly provided by Brett Saunders. (It should be emphasised that the code was used only
for this derivation and not copied otherwise. The implementation in this project is based
on the following derivation, not the original code.)

We start by defining a function $\mathbf{C}$ which evaluates to zero (or the null vector) for
all states which satisfy the constraint, and any non-zero value for all other states. This
function must be twice differentiable. We continue by calculating the first and second
derivatives with respect to time, $\dot{\mathbf{C}}$ and $\ddot{\mathbf{C}}$.

For any sort of valid constraint, we should be able to massage both $\dot{\mathbf{C}}$ and
$\ddot{\mathbf{C}}$ into a sum of products form. Moreover, in each product in
$\dot{\mathbf{C}}$, at least one of the variables should be either the linear velocity of
the centre of mass of one of the bodies, or the angular velocity of one of the bodies.
Use algebra to make this `chosen variable' the rightmost variable in each product, and
evaluate the rest of the product to a single matrix.

Now observe equation~\ref{cDotAndCDDot}. We can easily factor our expression for
$\dot{\mathbf{C}}$ into $\mathbf{J}$ and $\dot{\mathbf{x}}$ (since the latter contains
the linear and angular velocities for all bodies). $\mathbf{J}$ has the same number of rows
as there are constraints, and $6n$ columns for a system of $n$ bodies. Each of the matrices
in our expression for $\dot{\mathbf{C}}$ also has the same number of rows as there are
constraints, and has 3 columns. All we need to do is to find the correct horizontal position
of each of these matrices in $\mathbf{J}$, depending on the location of the `chosen variable'
in $\dot{\mathbf{x}}$. Thus we obtain the matrix $\mathbf{J}$.

Observe equation~\ref{cDotAndCDDot} again. Now we know $\ddot{\mathbf{C}}$, $\dot{\mathbf{x}}$,
$\mathbf{J}$ and $\ddot{\mathbf{x}}$~-- no problem. Evaluate
$\ddot{\mathbf{C}} - \mathbf{J}\ddot{\mathbf{x}}$, remembering that $\ddot{\mathbf{x}}$ is
the vector of linear accelerations and angular accelerations. The result should be an
expression which we can bring into just the same kind of sum of products form as we previously
did with $\dot{\mathbf{C}}$. By exactly the same procedure we factor out the linear velocities
and angular velocities (not accelerations!), thus obtaining $\dot{\mathbf{J}}$.

In case you were in doubt, this procedure is of course executed by a human on paper prior to
implementation. The simulation program will just plug numbers into the hard-coded expressions
for $\mathbf{C}$, $\dot{\mathbf{C}}$, $\mathbf{J}$ and $\dot{\mathbf{J}}$ during each time step
of the simulation.

Let us consider some examples of constraints to clarify the procedure.

\subsubsection{Example 1: Fixed point in space}

Let us implement a constraint which `nails' a particular point in a rigid body to a fixed point
in world space. (It's a very flexible nail, because despite fixing the position, it allows all
three modes of rotation.) Let $\mathbf{p}$ be the position of the centre of mass of our rigid
body, $\mathbf{s}$ the vector from the centre of mass to the point in the body we want to attach,
$\bm{\omega}$ the angular velocity of the rigid body, and $\mathbf{t}$ the coordinates in world
space that we want to nail the point to. Then we can set up a very simple constraint function,
\begin{equation}
\mathbf{C} = \mathbf{p} + \mathbf{s} - \mathbf{t}
\end{equation}
which equals the null vector when $\mathbf{p}+\mathbf{s}$ and $\mathbf{t}$ coincide, as required.
Since this is a three-dimensional vector equation, we are actually defining three constraints
at once. Since $\mathbf{t}$ does not change over time, we obtain
\begin{eqnarray}
\dot{\mathbf{C}} &=& \dot{\mathbf{p}} + \bm{\omega}\times\mathbf{s} \nonumber\\
&=& \dot{\mathbf{p}} - \mathbf{s}^*\,\bm{\omega} \\
\ddot{\mathbf{C}} &=& \ddot{\mathbf{p}} + \dot{\bm{\omega}}\times\mathbf{s} +
    \bm{\omega}\times(\bm{\omega}\times\mathbf{s}) \nonumber\\
&=& \ddot{\mathbf{p}} - \mathbf{s}^*\,\dot{\bm{\omega}} -
    (\bm{\omega}\times\mathbf{s})^*\,\bm{\omega}
\end{eqnarray}
(cf. similar derivations in~\cite{Kalra:95}). We have already moved the `chosen variables' to
the rightmost position of each product. We will now factor $\dot{\mathbf{C}}$ and write out the
components of $\mathbf{J}$ in terms of the vector components:

\begin{equation}
\label{constrEx1J}
\dot{\mathbf{C}} = \left[\begin{array}{ccc} 1&0&0\\0&1&0\\0&0&1 \end{array}\right]
    \dot{\mathbf{p}} + \left[\begin{array}{ccc}
    0 & s_3 & -s_2 \\ -s_3 & 0 & s_1 \\ s_2 & -s_1 & 0
    \end{array}\right] \bm{\omega}
\end{equation}

The two matrices in equation~\ref{constrEx1J} thus form two slices of $\mathbf{J}$ at
the locations appropriate for $\dot{\mathbf{p}}$ and $\bm{\omega}$. We now continue to the
next step of the procedure:

\begin{eqnarray}
\dot{\mathbf{J}}\dot{\mathbf{x}} = 
\ddot{\mathbf{C}} - \mathbf{J}\ddot{\mathbf{x}} &=&
    \big(\ddot{\mathbf{p}} - \mathbf{s}^*\,\dot{\bm{\omega}} -
    (\bm{\omega}\times\mathbf{s})^*\,\bm{\omega}\big) -
    (\ddot{\mathbf{p}} - \mathbf{s}^*\,\dot{\bm{\omega}}) \nonumber\\
& = & -(\bm{\omega}\times\mathbf{s})^*\,\bm{\omega} \nonumber\\
& = & \left[\begin{array}{ccc} 0 &
    \omega_1 s_2 - \omega_2 s_1 &
    \omega_1 s_3 - \omega_3 s_1 \\
    \omega_2 s_1 - \omega_1 s_2 & 0 &
    \omega_2 s_3 - \omega_3 s_2 \\
    \omega_3 s_1 - \omega_1 s_3 &
    \omega_3 s_2 - \omega_2 s_3 & 0
    \end{array}\right] \bm{\omega} \nonumber\\\label{constrEx1JDot}
\end{eqnarray}

We see that here there is only one slice for $\dot{\mathbf{J}}$; the rest of the matrix
is zero. Since $\bm{\omega}$ also occurs inside the matrix, there are actually several
alternative representations of this matrix which are equally valid.

\subsubsection{Example 2: Ball-and-socket joint}

Two rigid bodies are attached together at a particular point in each of the bodies.
They may not separate, but all three rotational degrees of freedom are permitted. This is
a good representation e.g. of a human shoulder joint.

Let $\mathbf{a}$ and $\mathbf{b}$ be the positions of the centres of mass in the first and
second rigid body respectively. Let $\mathbf{s}$ be the vector from $\mathbf{a}$ to the 
attachment point, and $\mathbf{t}$ the vector from $\mathbf{b}$ to the attachment point.
Also let $\bm{\omega}$ be the angular velocity of the first body, and $\bm{\phi}$ that
of the second. Then our constraint function and its derivatives are:

\begin{eqnarray}
\mathbf{C} &=& \mathbf{a} + \mathbf{s} - \mathbf{b} - \mathbf{t} \\
\dot{\mathbf{C}} &=& \dot{\mathbf{a}} + \bm{\omega}\times\mathbf{s} -
    \dot{\mathbf{b}} - \bm{\phi}\times\mathbf{t} \\
\ddot{\mathbf{C}} &=& \ddot{\mathbf{a}} + \dot{\bm{\omega}}\times\mathbf{s} +
    \bm{\omega}\times(\bm{\omega}\times\mathbf{s}) -
    \ddot{\mathbf{b}} - \dot{\bm{\phi}}\times\mathbf{t} -
    \bm{\phi}\times(\bm{\phi}\times\mathbf{t})
\end{eqnarray}

The rest of the derivation is very similar to the previous example. We obtain four
matrix slices for $\mathbf{J}$ and two slices for $\dot{\mathbf{J}}$.


\subsubsection{Example 3: Confinement to a plane}

A particular point in a rigid body is constrained to always lie within a particular plane.
It may move arbitrarily within that plane, and the body may rotate about the constrained point.

Let $\mathbf{b}$ be some arbitrary point inside the plane, $\mathbf{n}$ the normal vector of
the plane, $\mathbf{p}$ the centre of mass of the rigid body, $\bm{\omega}$ its angular
velocity, and $\mathbf{s}$ the vector from $\mathbf{p}$ to the point in the body at which the
constraint applies. Now the plane is specified as the set of all points $\mathbf{r}$ which
satisfy $(\mathbf{r} - \mathbf{b})\cdot\mathbf{n} = 0$.

This time, we lose only one degree of freedom, so our constraint function is a scalar. For
consistency's sake, we will write it as a one-row vector.

\begin{eqnarray}
\mathbf{C} &=& \mathbf{n}\cdot(\mathbf{p} + \mathbf{s} - \mathbf{b}) \nonumber\\
&=& \mathbf{n}\cdot\mathbf{p} + \mathbf{n}\cdot\mathbf{s} - \mathbf{n}\cdot\mathbf{b} \\
\dot{\mathbf{C}} & = & \mathbf{n}\cdot\dot{\mathbf{p}} +
    \mathbf{n}\cdot(\bm{\omega}\times\mathbf{s}) \nonumber\\
&=& \mathbf{n}^T\dot{\mathbf{p}} - \mathbf{n}^T \mathbf{s}^*\,\bm{\omega} \\
\ddot{\mathbf{C}} &=& \mathbf{n}\cdot\ddot{\mathbf{p}} +
    \mathbf{n}\cdot(\dot{\bm{\omega}}\times\mathbf{s}) +
    \mathbf{n}\cdot\big(\bm{\omega}\times(\bm{\omega}\times\mathbf{s})\big) \nonumber\\
&=& \mathbf{n}^T\ddot{\mathbf{p}} - \mathbf{n}^T \mathbf{s}^*\,\dot{\bm{\omega}} -
    \mathbf{n}^T (\bm{\omega}\times\mathbf{s})^*\,\bm{\omega}
\end{eqnarray}

Writing out the components of $\dot{\mathbf{C}}$:
\begin{equation}
\dot{\mathbf{C}} = \left[\begin{array}{ccc} n_1 & n_2 & n_3 \end{array}\right] \dot{\mathbf{p}} +
    \left[\begin{array}{c}
    n_3 s_2 - n_2 s_3 \\
    n_1 s_3 - n_3 s_1 \\
    n_2 s_1 - n_1 s_2 \end{array}\right]^T \bm{\omega}
\end{equation}
(the second slice of $\mathbf{J}$ has been written transposed for better readability). Again
we proceed as before:

\begin{eqnarray}
\dot{\mathbf{J}}\dot{\mathbf{x}} & = & \ddot{\mathbf{C}} - \mathbf{J}\ddot{\mathbf{x}} \nonumber\\
&=& -\mathbf{n}^T (\bm{\omega}\times\mathbf{s})^*\,\bm{\omega} \nonumber\\
&=& \left[\begin{array}{c}
    n_3 \omega_1 s_2 - n_3 \omega_2 s_1 - n_2 \omega_1 s_2 + n_2 \omega_2 s_1 \\
    n_1 \omega_1 s_2 - n_1 \omega_2 s_1 - n_3 \omega_2 s_3 + n_3 \omega_3 s_2 \\
    n_2 \omega_2 s_3 - n_2 \omega_3 s_2 - n_1 \omega_3 s_1 + n_1 \omega_1 s_3
    \end{array}\right]^T \bm{\omega}
\end{eqnarray}
Again note the transposition~-- in this example, $\mathbf{J}$ and $\dot{\mathbf{J}}$ only
have one row.
