\section{Successes}

In this project I have successfully created a general-purpose application which simulates
the dynamic behaviour of many different mechanical systems. In particular, it handles all
features necessary to animate articulated characters like humans or animals: an
arbitrary-shaped exterior, a skeleton with many different types of joints, and
collisions with other bodies. I have quantitatively verified the accuracy of the simulation
results for simple mechanical systems, and demonstrated that the program produces
realistic-looking animations in more complicated scenes involving a
model of a human character.

All requirements given in the project proposal have been met, except for the simulation of friction,
which was ruled out at an early stage because it is too complicated. All other features were
realized at a high quality standard. The implementation is carefully designed to use refined
algorithms which 
\begin{itemize}
\item keep numerical errors very small, doubtlessly within the bounds acceptable even for
    ambitious computer graphics purposes;
\item give the simulation a high degree of stability and reliability;
\item allow a reasonable run-time performance which can be traded off against the required
    accuracy by adjusting parameters.
\end{itemize}

The technology of this project is close to current research, and carrying it out
required some algorithms which I could not find in any of the publications I searched. I therefore
had to newly invent several algorithms, which took several weeks longer than anticipated in
the original schedule. However, I succeeded in adapting my management strategy
to the uncertainties facing this project, so that it could be completed in an organized way.

\section{Limitations}

There are only few explicit approximations in my simulation algorithms; this means that if they
were to be run using arbitrary-precision arithmetic and arbitrarily small time steps, the exact
physical behaviour would be obtained. The exceptions are:
\begin{itemize}
\item Given a triangle mesh and the density of a body, I currently approximate its mass and moment
    of inertia using a simple heuristic. For a better simulation, this should be
    replaced by an exact analysis of the geometric properties~\cite{Mirtich:96}.
\item Simple collisions (vertex/face and edge/edge cases) can be handled in an exact way, but
    more complicated collision geometries are currently approximated by planes or bounding
    spheres. This produces plausible-looking but slightly incorrect animations. Ideally,
    exact handling of any sort of collision would be desirable, but I am unsure whether such
    an algorithm exists.
\item As mentioned previously, static and sliding friction have been completely ignored.
    Algorithms to simulate these exist, but they are complicated~\cite{Baraff:PhD}.
\end{itemize}

Even with these added features, my program would not be a ready-to-market product. It particularly
lacks user-friendliness: there are several simulation parameters (tolerances and error thresholds)
which need to be configured differently depending on the situation in order to obtain good
results. Ideally the program should be able to determine these automatically. A higher execution
speed would also be beneficial, since the current cost~-- in my test cases, up to an hour of CPU
time per second of simulation time~-- limits productivity.

\section{Summary and outlook}

In summary, I consider this project a clear success. I am glad to have chosen such an ambitious
project because it was personally rewarding: in particular, the lack of existing algorithms
forced me to contemplate the background theory in great detail, which has given me a good
understanding of the subject area.

There are not many decisions that I would have taken differently in retrospect. In the schedule I
should have left more time for understanding the theory; I had thought there would be a simple,
``obvious'' way of realizing articulated bodies, rather than having to deal with constraints. This
meant I massively underestimated the problem. I would have also avoided the design mistake
mentioned in section~\ref{algorithmImplementation}.

The application is not yet user-friendly enough for general use, but with some additional effort
it might gain this potential. I can envisage working on it beyond the frame of this project,
and will consider the possibility of publishing some of the new algorithms to a wider audience.

I would like to thank Dr.~Brett Saunders~\cite{Saunders} and Dr.~Neil Dodgson (supervisor) for
their ideas and discussions, particularly in the early phase of this project.

\vspace{20pt}
\centerline{\texttt{http://www.maniation.com/}}
